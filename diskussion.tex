\section{Discussion}
\label{sec:Diskussion}

The processes of the measurements are not ideal. The temperature of the sample and of the recipient should equal all the time during the experiment. 
As the recipient heating system is not connected to the heating system of the sample, rather it is set manual. This obviously leads to a big 
amount of error. If the temperatures are not the same, there will be losses due to heat radiation.  Furthermore, the vacuum in the recipient 
cannot be maintained perfectly. Therefore, we get heat exchange by convection.\\
\\
The molar heat depends on the sample quantity and the temperature. Therefore, we compare our measured molar heat $C_p = (22.5\pm0.8)\,\mathrm{\frac{J}{mol\, K}}$ at room temperature 
with the literature.
The theoretical value (\cite{Cp}) is  
\begin{equation*}
    C_{p,theo} = 24.42 \,\mathrm{\frac{J}{mol\, K}} \,\, .
\end{equation*}
The deviation is $4.8 \%$. As we measured only up to $10.26 \,\unit{\celsius}$ the error is a little bit bigger because the molar heat would 
rise a bit more towards higher temperatures. We did not compare the same temperature of our measurement and literature. \\
\\
In \autoref{sec:debye} we found different values for the Debye temperature
\begin{align*}
    \symup{\theta_{D,exp}} &= (269.18 \pm  0.17)\,\unit{\kelvin} \\
    \symup{\theta_{D,theo}} &= 330.93\, \unit{\kelvin} \,\, .
\end{align*}
We can compare these with each other or with the literature. If we compare the theoretical value with the experimental we get a deviation  of 
$22.86\%$. 
If we compare it with the literature $\symup{\theta_D} = 343.5\,\unit{\kelvin}$ \cite{theta}, we get deviations of 
\begin{align*}
    \Delta \symup{\theta_{D,exp}} &= 27.53\% \\
    \Delta \symup{\theta_{D,theo}} &= 3.80 \% \,\, .
\end{align*}
\\
As the deviations are small, we can tell that the Debye model is a good approximation of the molar heat. Specifically, the molar heat $C_p$ has a small 
deviation. The a bit bigger deviation of the measured Debye temperature could be caused by reading off the quotient $\frac{\symup{\theta_D}}{T}$ from 
the manual \cite{ap47}. The coefficient is rounded a lot. 