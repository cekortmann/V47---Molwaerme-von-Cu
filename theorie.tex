\section{Theory}
\label{sec:theory}

The molar heat describes the quantity of heat, that is needed to warm one Mol by one Kelvin. This is characterized by the formula 
\begin{equation*}
    C = \frac{\Delta Q}{\Delta T} \,\, . 
\end{equation*}
The molar heat can be reduced to the pressure or to the volume. In this experiment we measure the molar heat with a constant pressure $C_{\symup{p}}$. 
For a constant volume a very high pressure would be needed. We can transform $C_{\symup{p}}$ to $C_{\symup{V}}$ by using 
\begin{equation*}
    C_{\symup{V}} = C_{\symup{p}} - 9{\alpha}^2\kappa V_{\symup{0}}T \,\, .
\end{equation*}
Here $\alpha$ describes the expansion coefficient and $\kappa$ the compressibilities.
The Debye temperature we mentioned before is material specific and depends on the phonon frequency. If the temperature is below this Debye temperature,
we have to consider quantum effects just as freezing of degrees of freedom. Above the Debye temperature all natural vibration states are occupied. 

\subsection{Classical theory of the molar heat}
\label{sec:classic}
In a solid, the energy is equally distributed on its degrees of freedom. A solid has six degrees of freedom, three for translation and three for rotation.
Each atom has an energy of $\frac{1}{2}\symup{k_B}T$ per degree of freedom. So the overall the energy is $E=3\symup{k_B}T$, where $\symup{k_B}$ 
is the Boltzmann constant. For one Mol we get 
\begin{equation*}
    E = 3\symup{k_B}\symup{N_A}T = \symup{R}T \,\, ,
\end{equation*}
where $\symup{N_A}$ is the Avogadro constant and $\symup{R}$ is the general gas constant. If we describe the specific heat capacity at a constant volume, 
we get the Dulong-Petit law
\begin{equation*}
    C_V = \left.\frac{\symup{\partial}U}{\symup{\partial}T}\right|_V=3 \symup{R} \,\, .
\end{equation*}
The Dulong-Petit law does not depend on material or temperature. At low temperatures measurements show no agreement. A classical description 
only approximates high temperatures ($T>> \symup{\theta_D}$).

\subsection{The Einstein model}
\label{sec:einstein}
The Einstein model takes quantum effects into account. It describes the solid as an oscillator. All oscillations are assumed to have the
uniform frequency $\omega_E$. Einstein considers the quantization of energy, which he assumes can only change in $\hbar \omega_E$. 
The Boltzmann distribution 
\begin{equation*}
    W(n)= \symup{exp}\left(-\frac{n\hbar \omega_E}{\symup{k_B}T}\right)
\end{equation*}
describes the probability that the oscillator has the energy $n\hbar\omega_E$ (n $\in$ N) at a given temperature. 
The mean energy is calculated by a summation and a normalization of all possible energies
\begin{equation*}
    \langle E \rangle = \frac{\hbar \omega_E}{\symup{exp} \left(\frac{\hbar \omega E}{\symup{k_B}T}\right)-1} \,\, .
\end{equation*}
With the Einstein temperature 
\begin{equation*}
    \symup{\theta_E} = \frac{\hbar \omega_E}{\symup{k_B}}
\end{equation*}
the molar heat at a constant volume can be depicted by 
\begin{equation*}
    C_V = 3\symup{N_A k_B}\left(\frac{\symup{\theta_E}}{T}\right)^2 \frac{\symup{e}^{\symup{\theta_E}/T}}{\left(\symup{e}^{\symup{\theta_E}/T}-1\right)^2} \,\, .
\end{equation*}
If we approximate the molar heat for very low and very high temperatures we get 
\begin{equation*}
    C_V = 
    \begin{cases}
        3 \symup{R}\left(\frac{\symup{\theta_E}}{T}\right)^2 \frac{\symup{e}^{\symup{\theta_E}/T}}{\left(\symup{e}^{\symup{\theta_E}/T}-1\right)^2} , & T << \symup{\theta_E} \\
        3 \symup{R} , T >> \symup{\theta_E}
    \end{cases}
    \, \, .
\end{equation*}
Again, for high temperatures we get the Dulong-Petit law, which approximates the experimental capacity good. For low temperatures we expect 
a $T^3$-dependence at the real measured values, that is not described by the Einstein model.

\subsection{The Debye model}
\label{sec:debye}
The Debye model introduces a spectral distribution $Z(\omega)$. With the Debye wave vector $q_D$ we get a linear dispersion
$\omega_i = v_i q_D$ for three branches $i$ with phase velocity $v_i$. For a crystal with a finite number of atoms $\symup{N_A}$ the 
number of oscillation frequencies is also finite. We get one frequency for each atom and each spatial direction, so that we have a total 
of $3\symup{N_A}$ oscillation frequencies. The surfaces of the same frequency are spherical surfaces with a volume of $\left(\frac{2\pi}{L}\right)^2$, 
so that we get the Debye wave vector 
\begin{equation*}
    q_D = \left(6\pi^2\frac{\symup{N_A}}{V}\right)^{1/3} \,\,. 
\end{equation*}
The Debye frequency is given by 
\begin{equation*}
    \omega_D = q_D v_i = v_i \left(6\pi^2 \frac{\symup{N_A}}{V}\right)^{1/3} \,\, .
\end{equation*}
The molar heat capacity at a constant volume can be calculated as 
\begin{equation*}
    C_V = 9\symup{N_A k_B}\left(\frac{T}{\symup{\theta_D}}\right)^3 \int_{0}^{\symup{\theta_D}/T}\frac{x^4\symup{e}^x}{\left(\symup{e}^x-1\right)^2}\, \symup{d}x \,\, .
\end{equation*}
Here the Debye temperature is defined as
\begin{equation*}
    \symup{\theta_D} = \frac{\hbar \omega_D}{\symup{k_B}} = \frac{\hbar v_s}{\symup{k_B}}\left(6\pi^2\frac{\symup{N_A}}{V}\right)^{1/3}
\end{equation*}
with the mean spead of sound $v_s$. \\
The approximation for very low and very high temperatures is given by 
\begin{equation*}
    C_V = 
    \begin{cases}
        \frac{12\pi^4}{5}\symup{R}\left(\frac{T}{\symup{\theta_D}}\right)^3 , & T << \symup{\theta_D} \\
        3\symup{R} , & T>> \symup{\theta_D}
    \end{cases}
    \,\, .
\end{equation*}
The dependence for high temperatures equals the Dulong-Petit law again. For low temperatures the $T^3$-dependence is correctly pictured by 
the Debye model.

\subsection{Calculation of $C_p$}
The molar heat at a constant pressure can be calculated by 
\begin{equation*}
    C_p = \frac{M}{m} \cdot \frac{E}{\Delta T} \,\, , 
\end{equation*}
where $M$ is the molecular weight and $m$ is the weight of the sample. 
The supplied energy $E$ is determined by 
\begin{equation*}
    E = U \cdot I \cdot \Delta t
\end{equation*}