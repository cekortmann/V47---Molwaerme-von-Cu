\section{Analysis of the experimental data}
\label{sec:Auswertung}

%\subsection{Fehlerrechnung}
\label{sec:Fehlerrechnung}
Für die Fehlerrechnung werden folgende Formeln aus der Vorlesung verwendet.
für den Mittelwert gilt
\begin{equation}
    \overline{x}=\frac{1}{N}\sum_{i=1}^N x_i ß\; \;\text{mit der Anzahl N und den Messwerten x} 
    \label{eqn:Mittelwert}
\end{equation}
Der Fehler für den Mittelwert lässt sich gemäß
\begin{equation}
    \increment \overline{x}=\frac{1}{\sqrt{N}}\sqrt{\frac{1}{N-1}\sum_{i=1}^N(x_i-\overline{x})^2}
    \label{eqn:FehlerMittelwert}
\end{equation}
berechnen.
Wenn im weiteren Verlauf der Berechnung mit der fehlerhaften Größe gerechnet wird, kann der Fehler der folgenden Größe
mittels Gaußscher Fehlerfortpflanzung berechnet werden. Die Formel hierfür ist
\begin{equation}
    \increment f= \sqrt{\sum_{i=1}^N\left(\frac{\partial f}{\partial x_i}\right)^2\cdot(\increment x_i)^2}.
    \label{eqn:GaussMittelwert}
\end{equation}

\subsection{Material-specific values}

The used copper sample has a weight of $m= 0.342 \, \unit{\kilo \gram}$.
In order to determine the molar heat capacity of copper, we need to know some of its characteristics.
So the important information are the molar volume $V_{\text{m,Cu}}$, the molar mass $M_{\text{Cu}}$, the density $\rho_{\text{Cu}}$ and 
the compression coefficient $\kappa_{\text{Cu}}$. 
The material-specific values can be found in \cite{6} as
\begin{align*}
    \rho_{\text{Cu}}&= 8920\,  \frac{\unit{\kilo \gram}}{\unit{\meter}^3} \; , \\
    V_{\text{m,Cu}}&=7.11 \cdot 10^{-6}\, \frac{\unit{\meter}^3}{\text{mol}} \; ,\\
    M_{\text{Cu}}&= V_{\text{m,Cu}} \cdot \rho_{\text{Cu}}= 0.0634\, \frac{\unit{\kilo \gram}}{\text{mol}}\; ,\\
    \kappa_{\text{Cu}}&= 140\, \unit{\giga \pascal}\; .
\end{align*}
In addition, the speed of sound in copper \cite{ap47} is needed for further calculations
\begin{align*}
    v_{\text{l}}&=4700\, \unit{\frac{\meter}{\second}}\; ,\\
v_{\text{t}}&=2260\, \unit{\frac{\meter}{\second}} \; .
\end{align*}

\subsection{Calculation of $C_{p}$}
The measurements to determine the heat capacity at constant pressure are displayed in \autoref{tab:Cp}.
To calculate the exact values, that are shown in the last column, \autoref{eqn:Cp} is used. 

\begin{table}
    \centering
    \caption{1}
    \begin{tabular}{c c c c c }
        \toprule
        $\Delta T \mathbin{/}\unit{\kelvin}$& $U\mathbin{/}\unit{\volt}$&  $I\mathbin{/}\unit{\ampere}$&$\Delta t \mathbin{/}\unit{\second}$& $C_p\mathbin{/}\unit{\frac{\joule}{\mol \kelvin}}$\\
        \midrule
5.68 \pm 0.33& 16.090 \pm 0.010& 0.15340 \pm 0.00010& 180.0 \pm 1.4& 14.5 \pm 0.9 \\
3.08 \pm 0.34& 16.110 \pm 0.010& 0.15360 \pm 0.00010& 129.0 \pm 1.4& 19.2 \pm 2.1\\
1.42 \pm 0.34& 15.920 \pm 0.010& 0.15190 \pm 0.00010& 51.0 \pm 1.4& 16.0 \pm 4.0\\
9.76 \pm 0.34& 16.070 \pm 0.010& 0.15300 \pm 0.00010& 370.0 \pm 1.4& 17.3 \pm 0.6\\
10.76 \pm 0.34& 16.160 \pm 0.010& 0.15360 \pm 0.00010& 430.0 \pm 1.4& 18.4 \pm 0.6\\
6.24 \pm 0.34& 16.190 \pm 0.010& 0.15380 \pm 0.00010& 265.0 \pm 1.4& 19.6 \pm 1.1\\
5.54 \pm 0.34& 16.220 \pm 0.010& 0.15400 \pm 0.00010& 249.0 \pm 1.4& 20.8 \pm 1.3\\
11.60 \pm 0.34& 16.270 \pm 0.010& 0.15430 \pm 0.00010& 516.0 \pm 1.4& 20.7 \pm 0.6\\
6.07 \pm 0.34& 16.290 \pm 0.010& 0.15440 \pm 0.00010& 264.0 \pm 1.4& 20.3 \pm 1.2\\
5.60 \pm 0.34& 16.300 \pm 0.010& 0.15450 \pm 0.00010& 243.0 \pm 1.4& 20.3 \pm 1.3\\
6.83 \pm 0.35& 16.310 \pm 0.010& 0.15470 \pm 0.00010& 299.0 \pm 1.4& 20.5 \pm 1.0\\
6.12 \pm 0.35& 16.330 \pm 0.010& 0.15470 \pm 0.00010& 292.0 \pm 1.4& 22.4 \pm 1.3\\
4.91 \pm 0.35& 16.340 \pm 0.010& 0.15480 \pm 0.00010& 318.0 \pm 1.4& 30.4 \pm 2.2\\
3.44 \pm 0.35& 16.340 \pm 0.010& 0.15480 \pm 0.00010& 102.0 \pm 1.4& 13.9 \pm 1.4\\
7.64 \pm 0.35& 16.360 \pm 0.010& 0.15460 \pm 0.00010& 334.0 \pm 1.4& 20.5 \pm 0.9\\
7.66 \pm 0.35& 16.370 \pm 0.010& 0.15500 \pm 0.00010& 327.0 \pm 1.4& 20.1 \pm 0.9\\
6.70 \pm 0.35& 16.370 \pm 0.010& 0.15500 \pm 0.00010& 343.0 \pm 1.4& 24.1 \pm 1.3\\
9.71 \pm 0.35& 16.380 \pm 0.010& 0.15510 \pm 0.00010& 513.0 \pm 1.4& 24.9 \pm 0.9\\
7.99 \pm 0.35& 16.380 \pm 0.010& 0.15510 \pm 0.00010& 387.0 \pm 1.4& 22.8 \pm 1.0\\
7.52 \pm 0.35& 16.390 \pm 0.010& 0.15520 \pm 0.00010& 368.0 \pm 1.4& 23.1 \pm 1.1\\
9.1 \pm 0.4& 16.390 \pm 0.010& 0.15520 \pm 0.00010& 449.0 \pm 1.4& 23.4 \pm 0.9\\
9.6 \pm 0.4& 16.390 \pm 0.010& 0.15530 \pm 0.00010& 579.0 \pm 1.4& 28.5 \pm 1.1\\
10.4 \pm 0.4& 16.390 \pm 0.010& 0.15530 \pm 0.00010& 432.0 \pm 1.4& 19.6 \pm 0.7\\
11.5 \pm 0.4& 16.390 \pm 0.010& 0.15540 \pm 0.00010& 594.0 \pm 1.4& 24.5 \pm 0.8\\
10.2 \pm 0.4& 16.380 \pm 0.010& 0.15530 \pm 0.00010& 507.0 \pm 1.4& 23.4 \pm 0.8\\
10.3 \pm 0.4& 16.380 \pm 0.010& 0.15530 \pm 0.00010& 491.0 \pm 1.4& 22.5 \pm 0.8\\
        \bottomrule
    \end{tabular}
    \label{tab:Differentiator}
\end{table}
\subsection{Calculation of $C_{V}$}
Coppers heat capacity at constant volume can be calculated by using \autoref{eqn:CV}. The values recorded this are shown in \autoref{tab:CV}.


\subsection{Calculation of the Debye temperature}% $\Theta _{\text{D}} $ }

When calculating the Debye temperature, only $C_{{V}}$ below $170\, \unit{\kelvin}$ are taken into account.
Then \cite{ap47} can be used to find the proportion $\frac{\Theta_{\text{D}}}{T}$.
By multiplying the temperature, a value for $\Theta _{\text{D}}$ can be found. The values of $C_V$, $\frac{\symup{\theta_D}}{T}$, $T$ and 
$\symup{\theta_D}$ are shown in \autoref{tab:debyetemp}.



The average leads to the final value of
\begin{equation*}
    \Theta _{\text{D}}= (269.18 \pm  0.17)\,\unit{\kelvin} \; .
\end{equation*}

In \autoref{fig:cv} the values aof $C_V$ are plotted in dependence of the temperature.
\begin{figure}
    \centering
        \includegraphics[width=0.5\textwidth]{build/cv.pdf}
        \caption{$C_V$ in dependence of the temperature.}
        \label{fig:cv}
\end{figure}

%\begin{figure}
 %   \centering
  %      \includegraphics[width=0.5\textwidth]{build/cv170.pdf}
   %     \caption{$C_V$ in dependence of the temperature up to $170\,\unit{\kelvin}$.}
    %    \label{fig:cv170}
%\end{figure}


The theoretical value for $\Theta _{\text{D}}$ can be determined by using \autoref{eqn:theoDebye}.
The number of particles is given by 
\begin{equation*}
    N=N_{\text{A}}\; \frac{m}{M}=3.076 \cdot 10^{24}\; .
\end{equation*}
By considering the sample's weight and copper's density the volume is given as 
\begin{equation*}
    V=\frac{m}{\rho}=3.632 \cdot 10^{-5} \, \unit{\meter}^3\; .
\end{equation*}
With these values, the Debye temperature results in
\begin{equation*}
    \Theta_{\text{D}}=330.93\, \unit{\kelvin}\; .
\end{equation*}